% Generated by Sphinx.
\def\sphinxdocclass{report}
\documentclass[letterpaper,10pt,english]{sphinxmanual}
\usepackage[utf8]{inputenc}
\DeclareUnicodeCharacter{00A0}{\nobreakspace}
\usepackage{cmap}
\usepackage[T1]{fontenc}
\usepackage{babel}
\usepackage{times}
\usepackage[Bjarne]{fncychap}
\usepackage{longtable}
\usepackage{sphinx}
\usepackage{multirow}


\title{AIETES Documentation}
\date{February 14, 2015}
\release{0.1}
\author{Andrew Bolster}
\newcommand{\sphinxlogo}{}
\renewcommand{\releasename}{Release}
\makeindex

\makeatletter
\def\PYG@reset{\let\PYG@it=\relax \let\PYG@bf=\relax%
    \let\PYG@ul=\relax \let\PYG@tc=\relax%
    \let\PYG@bc=\relax \let\PYG@ff=\relax}
\def\PYG@tok#1{\csname PYG@tok@#1\endcsname}
\def\PYG@toks#1+{\ifx\relax#1\empty\else%
    \PYG@tok{#1}\expandafter\PYG@toks\fi}
\def\PYG@do#1{\PYG@bc{\PYG@tc{\PYG@ul{%
    \PYG@it{\PYG@bf{\PYG@ff{#1}}}}}}}
\def\PYG#1#2{\PYG@reset\PYG@toks#1+\relax+\PYG@do{#2}}

\expandafter\def\csname PYG@tok@gd\endcsname{\def\PYG@tc##1{\textcolor[rgb]{0.63,0.00,0.00}{##1}}}
\expandafter\def\csname PYG@tok@gu\endcsname{\let\PYG@bf=\textbf\def\PYG@tc##1{\textcolor[rgb]{0.50,0.00,0.50}{##1}}}
\expandafter\def\csname PYG@tok@gt\endcsname{\def\PYG@tc##1{\textcolor[rgb]{0.00,0.27,0.87}{##1}}}
\expandafter\def\csname PYG@tok@gs\endcsname{\let\PYG@bf=\textbf}
\expandafter\def\csname PYG@tok@gr\endcsname{\def\PYG@tc##1{\textcolor[rgb]{1.00,0.00,0.00}{##1}}}
\expandafter\def\csname PYG@tok@cm\endcsname{\let\PYG@it=\textit\def\PYG@tc##1{\textcolor[rgb]{0.25,0.50,0.56}{##1}}}
\expandafter\def\csname PYG@tok@vg\endcsname{\def\PYG@tc##1{\textcolor[rgb]{0.73,0.38,0.84}{##1}}}
\expandafter\def\csname PYG@tok@m\endcsname{\def\PYG@tc##1{\textcolor[rgb]{0.13,0.50,0.31}{##1}}}
\expandafter\def\csname PYG@tok@mh\endcsname{\def\PYG@tc##1{\textcolor[rgb]{0.13,0.50,0.31}{##1}}}
\expandafter\def\csname PYG@tok@cs\endcsname{\def\PYG@tc##1{\textcolor[rgb]{0.25,0.50,0.56}{##1}}\def\PYG@bc##1{\setlength{\fboxsep}{0pt}\colorbox[rgb]{1.00,0.94,0.94}{\strut ##1}}}
\expandafter\def\csname PYG@tok@ge\endcsname{\let\PYG@it=\textit}
\expandafter\def\csname PYG@tok@vc\endcsname{\def\PYG@tc##1{\textcolor[rgb]{0.73,0.38,0.84}{##1}}}
\expandafter\def\csname PYG@tok@il\endcsname{\def\PYG@tc##1{\textcolor[rgb]{0.13,0.50,0.31}{##1}}}
\expandafter\def\csname PYG@tok@go\endcsname{\def\PYG@tc##1{\textcolor[rgb]{0.20,0.20,0.20}{##1}}}
\expandafter\def\csname PYG@tok@cp\endcsname{\def\PYG@tc##1{\textcolor[rgb]{0.00,0.44,0.13}{##1}}}
\expandafter\def\csname PYG@tok@gi\endcsname{\def\PYG@tc##1{\textcolor[rgb]{0.00,0.63,0.00}{##1}}}
\expandafter\def\csname PYG@tok@gh\endcsname{\let\PYG@bf=\textbf\def\PYG@tc##1{\textcolor[rgb]{0.00,0.00,0.50}{##1}}}
\expandafter\def\csname PYG@tok@ni\endcsname{\let\PYG@bf=\textbf\def\PYG@tc##1{\textcolor[rgb]{0.84,0.33,0.22}{##1}}}
\expandafter\def\csname PYG@tok@nl\endcsname{\let\PYG@bf=\textbf\def\PYG@tc##1{\textcolor[rgb]{0.00,0.13,0.44}{##1}}}
\expandafter\def\csname PYG@tok@nn\endcsname{\let\PYG@bf=\textbf\def\PYG@tc##1{\textcolor[rgb]{0.05,0.52,0.71}{##1}}}
\expandafter\def\csname PYG@tok@no\endcsname{\def\PYG@tc##1{\textcolor[rgb]{0.38,0.68,0.84}{##1}}}
\expandafter\def\csname PYG@tok@na\endcsname{\def\PYG@tc##1{\textcolor[rgb]{0.25,0.44,0.63}{##1}}}
\expandafter\def\csname PYG@tok@nb\endcsname{\def\PYG@tc##1{\textcolor[rgb]{0.00,0.44,0.13}{##1}}}
\expandafter\def\csname PYG@tok@nc\endcsname{\let\PYG@bf=\textbf\def\PYG@tc##1{\textcolor[rgb]{0.05,0.52,0.71}{##1}}}
\expandafter\def\csname PYG@tok@nd\endcsname{\let\PYG@bf=\textbf\def\PYG@tc##1{\textcolor[rgb]{0.33,0.33,0.33}{##1}}}
\expandafter\def\csname PYG@tok@ne\endcsname{\def\PYG@tc##1{\textcolor[rgb]{0.00,0.44,0.13}{##1}}}
\expandafter\def\csname PYG@tok@nf\endcsname{\def\PYG@tc##1{\textcolor[rgb]{0.02,0.16,0.49}{##1}}}
\expandafter\def\csname PYG@tok@si\endcsname{\let\PYG@it=\textit\def\PYG@tc##1{\textcolor[rgb]{0.44,0.63,0.82}{##1}}}
\expandafter\def\csname PYG@tok@s2\endcsname{\def\PYG@tc##1{\textcolor[rgb]{0.25,0.44,0.63}{##1}}}
\expandafter\def\csname PYG@tok@vi\endcsname{\def\PYG@tc##1{\textcolor[rgb]{0.73,0.38,0.84}{##1}}}
\expandafter\def\csname PYG@tok@nt\endcsname{\let\PYG@bf=\textbf\def\PYG@tc##1{\textcolor[rgb]{0.02,0.16,0.45}{##1}}}
\expandafter\def\csname PYG@tok@nv\endcsname{\def\PYG@tc##1{\textcolor[rgb]{0.73,0.38,0.84}{##1}}}
\expandafter\def\csname PYG@tok@s1\endcsname{\def\PYG@tc##1{\textcolor[rgb]{0.25,0.44,0.63}{##1}}}
\expandafter\def\csname PYG@tok@gp\endcsname{\let\PYG@bf=\textbf\def\PYG@tc##1{\textcolor[rgb]{0.78,0.36,0.04}{##1}}}
\expandafter\def\csname PYG@tok@sh\endcsname{\def\PYG@tc##1{\textcolor[rgb]{0.25,0.44,0.63}{##1}}}
\expandafter\def\csname PYG@tok@ow\endcsname{\let\PYG@bf=\textbf\def\PYG@tc##1{\textcolor[rgb]{0.00,0.44,0.13}{##1}}}
\expandafter\def\csname PYG@tok@sx\endcsname{\def\PYG@tc##1{\textcolor[rgb]{0.78,0.36,0.04}{##1}}}
\expandafter\def\csname PYG@tok@bp\endcsname{\def\PYG@tc##1{\textcolor[rgb]{0.00,0.44,0.13}{##1}}}
\expandafter\def\csname PYG@tok@c1\endcsname{\let\PYG@it=\textit\def\PYG@tc##1{\textcolor[rgb]{0.25,0.50,0.56}{##1}}}
\expandafter\def\csname PYG@tok@kc\endcsname{\let\PYG@bf=\textbf\def\PYG@tc##1{\textcolor[rgb]{0.00,0.44,0.13}{##1}}}
\expandafter\def\csname PYG@tok@c\endcsname{\let\PYG@it=\textit\def\PYG@tc##1{\textcolor[rgb]{0.25,0.50,0.56}{##1}}}
\expandafter\def\csname PYG@tok@mf\endcsname{\def\PYG@tc##1{\textcolor[rgb]{0.13,0.50,0.31}{##1}}}
\expandafter\def\csname PYG@tok@err\endcsname{\def\PYG@bc##1{\setlength{\fboxsep}{0pt}\fcolorbox[rgb]{1.00,0.00,0.00}{1,1,1}{\strut ##1}}}
\expandafter\def\csname PYG@tok@mb\endcsname{\def\PYG@tc##1{\textcolor[rgb]{0.13,0.50,0.31}{##1}}}
\expandafter\def\csname PYG@tok@ss\endcsname{\def\PYG@tc##1{\textcolor[rgb]{0.32,0.47,0.09}{##1}}}
\expandafter\def\csname PYG@tok@sr\endcsname{\def\PYG@tc##1{\textcolor[rgb]{0.14,0.33,0.53}{##1}}}
\expandafter\def\csname PYG@tok@mo\endcsname{\def\PYG@tc##1{\textcolor[rgb]{0.13,0.50,0.31}{##1}}}
\expandafter\def\csname PYG@tok@kd\endcsname{\let\PYG@bf=\textbf\def\PYG@tc##1{\textcolor[rgb]{0.00,0.44,0.13}{##1}}}
\expandafter\def\csname PYG@tok@mi\endcsname{\def\PYG@tc##1{\textcolor[rgb]{0.13,0.50,0.31}{##1}}}
\expandafter\def\csname PYG@tok@kn\endcsname{\let\PYG@bf=\textbf\def\PYG@tc##1{\textcolor[rgb]{0.00,0.44,0.13}{##1}}}
\expandafter\def\csname PYG@tok@o\endcsname{\def\PYG@tc##1{\textcolor[rgb]{0.40,0.40,0.40}{##1}}}
\expandafter\def\csname PYG@tok@kr\endcsname{\let\PYG@bf=\textbf\def\PYG@tc##1{\textcolor[rgb]{0.00,0.44,0.13}{##1}}}
\expandafter\def\csname PYG@tok@s\endcsname{\def\PYG@tc##1{\textcolor[rgb]{0.25,0.44,0.63}{##1}}}
\expandafter\def\csname PYG@tok@kp\endcsname{\def\PYG@tc##1{\textcolor[rgb]{0.00,0.44,0.13}{##1}}}
\expandafter\def\csname PYG@tok@w\endcsname{\def\PYG@tc##1{\textcolor[rgb]{0.73,0.73,0.73}{##1}}}
\expandafter\def\csname PYG@tok@kt\endcsname{\def\PYG@tc##1{\textcolor[rgb]{0.56,0.13,0.00}{##1}}}
\expandafter\def\csname PYG@tok@sc\endcsname{\def\PYG@tc##1{\textcolor[rgb]{0.25,0.44,0.63}{##1}}}
\expandafter\def\csname PYG@tok@sb\endcsname{\def\PYG@tc##1{\textcolor[rgb]{0.25,0.44,0.63}{##1}}}
\expandafter\def\csname PYG@tok@k\endcsname{\let\PYG@bf=\textbf\def\PYG@tc##1{\textcolor[rgb]{0.00,0.44,0.13}{##1}}}
\expandafter\def\csname PYG@tok@se\endcsname{\let\PYG@bf=\textbf\def\PYG@tc##1{\textcolor[rgb]{0.25,0.44,0.63}{##1}}}
\expandafter\def\csname PYG@tok@sd\endcsname{\let\PYG@it=\textit\def\PYG@tc##1{\textcolor[rgb]{0.25,0.44,0.63}{##1}}}

\def\PYGZbs{\char`\\}
\def\PYGZus{\char`\_}
\def\PYGZob{\char`\{}
\def\PYGZcb{\char`\}}
\def\PYGZca{\char`\^}
\def\PYGZam{\char`\&}
\def\PYGZlt{\char`\<}
\def\PYGZgt{\char`\>}
\def\PYGZsh{\char`\#}
\def\PYGZpc{\char`\%}
\def\PYGZdl{\char`\$}
\def\PYGZhy{\char`\-}
\def\PYGZsq{\char`\'}
\def\PYGZdq{\char`\"}
\def\PYGZti{\char`\~}
% for compatibility with earlier versions
\def\PYGZat{@}
\def\PYGZlb{[}
\def\PYGZrb{]}
\makeatother

\renewcommand\PYGZsq{\textquotesingle}

\begin{document}

\maketitle
\tableofcontents
\phantomsection\label{index::doc}


Contents:
\phantomsection\label{index:module-aietes}\index{aietes (module)}\begin{itemize}
\item {} 
This file is part of the Aietes Framework (\href{https://github.com/andrewbolster/aietes}{https://github.com/andrewbolster/aietes})

\item {} 
\item {} \begin{enumerate}
\setcounter{enumi}{2}
\item {} 
Copyright 2013 Andrew Bolster (\href{http://andrewbolster.info/}{http://andrewbolster.info/}) and others.

\end{enumerate}

\item {} 
\item {} 
All rights reserved. This program and the accompanying materials

\item {} 
are made available under the terms of the Eclipse Public License v1.0

\item {} 
which accompanies this distribution, and is available at

\item {} 
\href{http://www.eclipse.org/legal/epl-v10.html}{http://www.eclipse.org/legal/epl-v10.html}

\item {} 
\item {} 
Contributors:

\item {} 
Andrew Bolster, Queen's University Belfast (-Aug 2013), University of Liverpool (Sept 2014-)

\end{itemize}
\index{Simulation (class in aietes)}

\begin{fulllineitems}
\phantomsection\label{index:aietes.Simulation}\pysiglinewithargsret{\strong{class }\code{aietes.}\bfcode{Simulation}}{\emph{*args}, \emph{**kwargs}}{}
Defines a single simulation
Keyword Arguments:
\begin{quote}

title:str(time)
progress\_display:bool(True)
working\_directory:str(/dev/shm)
logtofile:str(None)
logtoconsole:logging.level(INFO)
logger:logging.logger(None)
config\_file:str(None)
config:dict(None)
\end{quote}
\index{configure\_environment() (aietes.Simulation method)}

\begin{fulllineitems}
\phantomsection\label{index:aietes.Simulation.configure_environment}\pysiglinewithargsret{\bfcode{configure\_environment}}{\emph{env\_config}}{}
Configure the physical environment within which the simulation executed
Assumes empty unless told otherwise
:param env\_config:

\end{fulllineitems}

\index{configure\_nodes() (aietes.Simulation method)}

\begin{fulllineitems}
\phantomsection\label{index:aietes.Simulation.configure_nodes}\pysiglinewithargsret{\bfcode{configure\_nodes}}{}{}
Configure `fleets' of nodes for simulation
Fleets are purely logical in this case

\end{fulllineitems}

\index{current\_state() (aietes.Simulation method)}

\begin{fulllineitems}
\phantomsection\label{index:aietes.Simulation.current_state}\pysiglinewithargsret{\bfcode{current\_state}}{}{}~\begin{quote}\begin{description}
\item[{Returns}] \leavevmode


\end{description}\end{quote}

\end{fulllineitems}

\index{delta\_t() (aietes.Simulation method)}

\begin{fulllineitems}
\phantomsection\label{index:aietes.Simulation.delta_t}\pysiglinewithargsret{\bfcode{delta\_t}}{\emph{now}, \emph{then}}{}
Time in seconds between two simulation times
:param now:
:param then:

\end{fulllineitems}

\index{generate\_a\_node() (aietes.Simulation method)}

\begin{fulllineitems}
\phantomsection\label{index:aietes.Simulation.generate_a_node}\pysiglinewithargsret{\bfcode{generate\_a\_node}}{\emph{node\_name}, \emph{node\_config}}{}
If a node is named in the configuration file, use the defined initial vector
otherwise, use configured behaviour to assign an initial vector
:param node\_name:
:param node\_config:

\end{fulllineitems}

\index{generate\_datapackage() (aietes.Simulation method)}

\begin{fulllineitems}
\phantomsection\label{index:aietes.Simulation.generate_datapackage}\pysiglinewithargsret{\bfcode{generate\_datapackage}}{\emph{*args}, \emph{**kwargs}}{}
Creates a bounos.DataPackage object from the current sim
:param args:
:param kwargs:

\end{fulllineitems}

\index{inner\_join() (aietes.Simulation method)}

\begin{fulllineitems}
\phantomsection\label{index:aietes.Simulation.inner_join}\pysiglinewithargsret{\bfcode{inner\_join}}{}{}
When all nodes have a move flag and none are processing

\end{fulllineitems}

\index{now() (aietes.Simulation static method)}

\begin{fulllineitems}
\phantomsection\label{index:aietes.Simulation.now}\pysiglinewithargsret{\strong{static }\bfcode{now}}{}{}~\begin{quote}\begin{description}
\item[{Returns}] \leavevmode


\end{description}\end{quote}

\end{fulllineitems}

\index{outer\_join() (aietes.Simulation method)}

\begin{fulllineitems}
\phantomsection\label{index:aietes.Simulation.outer_join}\pysiglinewithargsret{\bfcode{outer\_join}}{}{}
When all nodes have a processing flag and none are moving

\end{fulllineitems}

\index{populate\_config() (aietes.Simulation class method)}

\begin{fulllineitems}
\phantomsection\label{index:aietes.Simulation.populate_config}\pysiglinewithargsret{\strong{classmethod }\bfcode{populate\_config}}{\emph{config}, \emph{retain\_default=False}}{}~\begin{quote}\begin{description}
\item[{Parameters}] \leavevmode\begin{itemize}
\item {} 
\textbf{config} -- 

\item {} 
\textbf{retain\_default} -- 

\end{itemize}

\item[{Returns}] \leavevmode
\begin{quote}\begin{description}
\item[{raise ConfigError}] \leavevmode
\end{description}\end{quote}


\end{description}\end{quote}

\end{fulllineitems}

\index{postprocess() (aietes.Simulation method)}

\begin{fulllineitems}
\phantomsection\label{index:aietes.Simulation.postprocess}\pysiglinewithargsret{\bfcode{postprocess}}{\emph{log=None}, \emph{output\_file=None}, \emph{output\_path=None}, \emph{display\_frames=None}, \emph{data\_file=False}, \emph{movie\_file=False}, \emph{gif=False}, \emph{input\_file=None}, \emph{plot=False}, \emph{xres=1024}, \emph{yres=768}, \emph{fps=24}, \emph{extent=True}}{}
Performs output and positions generation for a given simulation
:param log:
:param output\_file:
:param output\_path:
:param display\_frames:
:param data\_file:
:param movie\_file:
:param gif:
:param input\_file:
:param plot:
:param xres:
:param yres:
:param fps:
:param extent:

\end{fulllineitems}

\index{prepare() (aietes.Simulation method)}

\begin{fulllineitems}
\phantomsection\label{index:aietes.Simulation.prepare}\pysiglinewithargsret{\bfcode{prepare}}{\emph{waits=False}, \emph{*args}, \emph{**kwargs}}{}~\begin{description}
\item[{Args:}] \leavevmode
waits(bool): set if running interactively (i.e. sim will wait for external actions)
sim\_time(int): override the simulation duration

\item[{Raises:}] \leavevmode
SystemExit on configuration error in setting log)

\item[{Returns:}] \leavevmode
Dict: \{`sim\_time':prepared sim duration (int),\}
This is an extensible interface that can be added to but must maintain compatibility.

\end{description}

\end{fulllineitems}

\index{reverse\_node\_lookup() (aietes.Simulation method)}

\begin{fulllineitems}
\phantomsection\label{index:aietes.Simulation.reverse_node_lookup}\pysiglinewithargsret{\bfcode{reverse\_node\_lookup}}{\emph{uuid}}{}
Return Node Given UUID
:param uuid:

\end{fulllineitems}

\index{simulate() (aietes.Simulation method)}

\begin{fulllineitems}
\phantomsection\label{index:aietes.Simulation.simulate}\pysiglinewithargsret{\bfcode{simulate}}{\emph{callback=None}}{}
Initiate the processed Simulation
:param callback:
Args:
\begin{quote}

callback(func): Callback function to be called at each execution step
\end{quote}
\begin{description}
\item[{Returns:}] \leavevmode
Simulation Duration in ticks (generally seconds)

\end{description}

\end{fulllineitems}


\end{fulllineitems}

\index{go() (in module aietes)}

\begin{fulllineitems}
\phantomsection\label{index:aietes.go}\pysiglinewithargsret{\code{aietes.}\bfcode{go}}{\emph{options}, \emph{args=None}}{}~\begin{quote}\begin{description}
\item[{Parameters}] \leavevmode\begin{itemize}
\item {} 
\textbf{options} -- 

\item {} 
\textbf{args} -- 

\end{itemize}

\item[{Returns}] \leavevmode


\end{description}\end{quote}

\end{fulllineitems}

\index{main() (in module aietes)}

\begin{fulllineitems}
\phantomsection\label{index:aietes.main}\pysiglinewithargsret{\code{aietes.}\bfcode{main}}{}{}
Everyone knows what the main does; it does everything!

\end{fulllineitems}

\index{option\_parser() (in module aietes)}

\begin{fulllineitems}
\phantomsection\label{index:aietes.option_parser}\pysiglinewithargsret{\code{aietes.}\bfcode{option\_parser}}{}{}~\begin{quote}\begin{description}
\item[{Returns}] \leavevmode


\end{description}\end{quote}

\end{fulllineitems}

\phantomsection\label{index:module-aietes.Tools}\index{aietes.Tools (module)}\begin{itemize}
\item {} 
This file is part of the Aietes Framework (\href{https://github.com/andrewbolster/aietes}{https://github.com/andrewbolster/aietes})

\item {} 
\item {} \begin{enumerate}
\setcounter{enumi}{2}
\item {} 
Copyright 2013 Andrew Bolster (\href{http://andrewbolster.info/}{http://andrewbolster.info/}) and others.

\end{enumerate}

\item {} 
\item {} 
All rights reserved. This program and the accompanying materials

\item {} 
are made available under the terms of the Eclipse Public License v1.0

\item {} 
which accompanies this distribution, and is available at

\item {} 
\href{http://www.eclipse.org/legal/epl-v10.html}{http://www.eclipse.org/legal/epl-v10.html}

\item {} 
\item {} 
Contributors:

\item {} 
Andrew Bolster, Queen's University Belfast (-Aug 2013), University of Liverpool (Sept 2014-)

\end{itemize}
\index{AutoSyncShelf (class in aietes.Tools)}

\begin{fulllineitems}
\phantomsection\label{index:aietes.Tools.AutoSyncShelf}\pysiglinewithargsret{\strong{class }\code{aietes.Tools.}\bfcode{AutoSyncShelf}}{\emph{filename}, \emph{protocol=2}, \emph{writeback=True}}{}~\begin{quote}\begin{description}
\item[{Parameters}] \leavevmode\begin{itemize}
\item {} 
\textbf{filename} -- 

\item {} 
\textbf{protocol} -- 

\item {} 
\textbf{writeback} -- 

\end{itemize}

\end{description}\end{quote}

\end{fulllineitems}

\index{ConfigError}

\begin{fulllineitems}
\phantomsection\label{index:aietes.Tools.ConfigError}\pysiglinewithargsret{\strong{exception }\code{aietes.Tools.}\bfcode{ConfigError}}{\emph{message}, \emph{errors=None}}{}
Raised when a configuration cannot be validated through ConfigObj/Validator
Contains a `status' with the boolean dict representation of the error

\end{fulllineitems}

\index{Dotdict (class in aietes.Tools)}

\begin{fulllineitems}
\phantomsection\label{index:aietes.Tools.Dotdict}\pysiglinewithargsret{\strong{class }\code{aietes.Tools.}\bfcode{Dotdict}}{\emph{arg}, \emph{**kwargs}}{}~\begin{quote}\begin{description}
\item[{Parameters}] \leavevmode\begin{itemize}
\item {} 
\textbf{arg} -- 

\item {} 
\textbf{kwargs} -- 

\end{itemize}

\end{description}\end{quote}

\end{fulllineitems}

\index{Dotdictify (class in aietes.Tools)}

\begin{fulllineitems}
\phantomsection\label{index:aietes.Tools.Dotdictify}\pysiglinewithargsret{\strong{class }\code{aietes.Tools.}\bfcode{Dotdictify}}{\emph{value=None}, \emph{**kwargs}}{}~\begin{quote}\begin{description}
\item[{Parameters}] \leavevmode\begin{itemize}
\item {} 
\textbf{value} -- 

\item {} 
\textbf{kwargs} -- 

\end{itemize}

\item[{Raises TypeError}] \leavevmode


\end{description}\end{quote}

\end{fulllineitems}

\index{KeepRefs (class in aietes.Tools)}

\begin{fulllineitems}
\phantomsection\label{index:aietes.Tools.KeepRefs}\pysigline{\strong{class }\code{aietes.Tools.}\bfcode{KeepRefs}}~\index{get\_instances() (aietes.Tools.KeepRefs class method)}

\begin{fulllineitems}
\phantomsection\label{index:aietes.Tools.KeepRefs.get_instances}\pysiglinewithargsret{\strong{classmethod }\bfcode{get\_instances}}{}{}
\end{fulllineitems}


\end{fulllineitems}

\index{MapEntry (class in aietes.Tools)}

\begin{fulllineitems}
\phantomsection\label{index:aietes.Tools.MapEntry}\pysiglinewithargsret{\strong{class }\code{aietes.Tools.}\bfcode{MapEntry}}{\emph{object\_id}, \emph{position}, \emph{velocity}, \emph{name=None}, \emph{distance=None}}{}~\begin{quote}\begin{description}
\item[{Parameters}] \leavevmode\begin{itemize}
\item {} 
\textbf{object\_id} -- 

\item {} 
\textbf{position} -- 

\item {} 
\textbf{velocity} -- 

\item {} 
\textbf{name} -- 

\item {} 
\textbf{distance} -- 

\end{itemize}

\end{description}\end{quote}

\end{fulllineitems}

\index{MemoryEntry (class in aietes.Tools)}

\begin{fulllineitems}
\phantomsection\label{index:aietes.Tools.MemoryEntry}\pysiglinewithargsret{\strong{class }\code{aietes.Tools.}\bfcode{MemoryEntry}}{\emph{object\_id}, \emph{position}, \emph{velocity}, \emph{distance=None}, \emph{name=None}}{}~\begin{quote}\begin{description}
\item[{Parameters}] \leavevmode\begin{itemize}
\item {} 
\textbf{object\_id} -- 

\item {} 
\textbf{position} -- 

\item {} 
\textbf{velocity} -- 

\item {} 
\textbf{distance} -- 

\item {} 
\textbf{name} -- 

\end{itemize}

\end{description}\end{quote}

\end{fulllineitems}

\index{SimTimeFilter (class in aietes.Tools)}

\begin{fulllineitems}
\phantomsection\label{index:aietes.Tools.SimTimeFilter}\pysiglinewithargsret{\strong{class }\code{aietes.Tools.}\bfcode{SimTimeFilter}}{\emph{name='`}}{}
Brings Sim.now() into usefulness

\end{fulllineitems}

\index{add\_ndarray\_to\_set() (in module aietes.Tools)}

\begin{fulllineitems}
\phantomsection\label{index:aietes.Tools.add_ndarray_to_set}\pysiglinewithargsret{\code{aietes.Tools.}\bfcode{add\_ndarray\_to\_set}}{\emph{ndarray}, \emph{list}}{}~\begin{quote}\begin{description}
\item[{Parameters}] \leavevmode\begin{itemize}
\item {} 
\textbf{ndarray} -- 

\item {} 
\textbf{list} -- 

\end{itemize}

\item[{Returns}] \leavevmode


\end{description}\end{quote}

\end{fulllineitems}

\index{agitate\_position() (in module aietes.Tools)}

\begin{fulllineitems}
\phantomsection\label{index:aietes.Tools.agitate_position}\pysiglinewithargsret{\code{aietes.Tools.}\bfcode{agitate\_position}}{\emph{position}, \emph{maximum}, \emph{var=10}, \emph{minimum=None}}{}
Fluff a position i by randn*var, limited to maximum/minimum
:param position:
:param maximum:
:param var:
:param minimum:
:return:

\end{fulllineitems}

\index{angle\_between() (in module aietes.Tools)}

\begin{fulllineitems}
\phantomsection\label{index:aietes.Tools.angle_between}\pysiglinewithargsret{\code{aietes.Tools.}\bfcode{angle\_between}}{\emph{v1}, \emph{v2}}{}
Returns the angle in radians between vectors `v1' and `v2':

\begin{Verbatim}[commandchars=\\\{\}]
\PYG{g+gp}{\PYGZgt{}\PYGZgt{}\PYGZgt{} }\PYG{n}{angle\PYGZus{}between}\PYG{p}{(}\PYG{p}{(}\PYG{l+m+mi}{1}\PYG{p}{,} \PYG{l+m+mi}{0}\PYG{p}{,} \PYG{l+m+mi}{0}\PYG{p}{)}\PYG{p}{,} \PYG{p}{(}\PYG{l+m+mi}{0}\PYG{p}{,} \PYG{l+m+mi}{1}\PYG{p}{,} \PYG{l+m+mi}{0}\PYG{p}{)}\PYG{p}{)}
\end{Verbatim}

1.5707963267948966
\textgreater{}\textgreater{}\textgreater{} angle\_between((1, 0, 0), (1, 0, 0))
0.0
\textgreater{}\textgreater{}\textgreater{} angle\_between((1, 0, 0), (-1, 0, 0))
3.1415926535897931

\end{fulllineitems}

\index{are\_equal\_waypoints() (in module aietes.Tools)}

\begin{fulllineitems}
\phantomsection\label{index:aietes.Tools.are_equal_waypoints}\pysiglinewithargsret{\code{aietes.Tools.}\bfcode{are\_equal\_waypoints}}{\emph{wps}}{}~\begin{description}
\item[{Compare Waypoint Objects as used by WaypointMixin ({[}pos{]},prox)}] \leavevmode
Will exclude `None' records in wps and only compare valid waypoint lists

\end{description}
\begin{quote}\begin{description}
\item[{Parameters}] \leavevmode
\textbf{wps} -- 

\end{description}\end{quote}

\end{fulllineitems}

\index{bearing() (in module aietes.Tools)}

\begin{fulllineitems}
\phantomsection\label{index:aietes.Tools.bearing}\pysiglinewithargsret{\code{aietes.Tools.}\bfcode{bearing}}{\emph{v}}{}
radian angle between a given vector and `north'
:param v:

\end{fulllineitems}

\index{db2linear() (in module aietes.Tools)}

\begin{fulllineitems}
\phantomsection\label{index:aietes.Tools.db2linear}\pysiglinewithargsret{\code{aietes.Tools.}\bfcode{db2linear}}{\emph{db}}{}~\begin{quote}\begin{description}
\item[{Parameters}] \leavevmode
\textbf{db} -- 

\item[{Returns}] \leavevmode


\end{description}\end{quote}

\end{fulllineitems}

\index{distance() (in module aietes.Tools)}

\begin{fulllineitems}
\phantomsection\label{index:aietes.Tools.distance}\pysiglinewithargsret{\code{aietes.Tools.}\bfcode{distance}}{\emph{pos\_a}, \emph{pos\_b}, \emph{scale=1}}{}
Return the distance between two positions
:param pos\_a:
:param pos\_b:
:param scale:

\end{fulllineitems}

\index{fudge\_normal() (in module aietes.Tools)}

\begin{fulllineitems}
\phantomsection\label{index:aietes.Tools.fudge_normal}\pysiglinewithargsret{\code{aietes.Tools.}\bfcode{fudge\_normal}}{\emph{value}, \emph{stdev}}{}~\begin{quote}\begin{description}
\item[{Parameters}] \leavevmode\begin{itemize}
\item {} 
\textbf{value} -- 

\item {} 
\textbf{stdev} -- 

\end{itemize}

\item[{Returns}] \leavevmode
\begin{quote}\begin{description}
\item[{raise ValueError}] \leavevmode
\end{description}\end{quote}


\end{description}\end{quote}

\end{fulllineitems}

\index{generate\_names() (in module aietes.Tools)}

\begin{fulllineitems}
\phantomsection\label{index:aietes.Tools.generate_names}\pysiglinewithargsret{\code{aietes.Tools.}\bfcode{generate\_names}}{\emph{count}, \emph{naming\_convention=None}, \emph{existing\_names=None}}{}~\begin{quote}\begin{description}
\item[{Parameters}] \leavevmode\begin{itemize}
\item {} 
\textbf{count} -- 

\item {} 
\textbf{naming\_convention} -- 

\item {} 
\textbf{existing\_names} -- 

\end{itemize}

\item[{Returns}] \leavevmode
\begin{quote}\begin{description}
\item[{raise ConfigError}] \leavevmode
\end{description}\end{quote}


\end{description}\end{quote}

\end{fulllineitems}

\index{get\_config() (in module aietes.Tools)}

\begin{fulllineitems}
\phantomsection\label{index:aietes.Tools.get_config}\pysiglinewithargsret{\code{aietes.Tools.}\bfcode{get\_config}}{\emph{source\_config=None}, \emph{config\_spec='/home/bolster/src/aietes/src/aietes/configs/default.conf'}}{}~\begin{description}
\item[{Get a configuration, either using default values from aietes.configs or}] \leavevmode
by taking a configobj compatible file path

\end{description}
\begin{quote}\begin{description}
\item[{Parameters}] \leavevmode\begin{itemize}
\item {} 
\textbf{source\_config} -- 

\item {} 
\textbf{config\_spec} -- 

\end{itemize}

\end{description}\end{quote}

\end{fulllineitems}

\index{get\_config\_file() (in module aietes.Tools)}

\begin{fulllineitems}
\phantomsection\label{index:aietes.Tools.get_config_file}\pysiglinewithargsret{\code{aietes.Tools.}\bfcode{get\_config\_file}}{\emph{config}}{}
Return the full path to a config of a given filename if its in the default config path
:param config:
:return:

\end{fulllineitems}

\index{get\_latest\_aietes\_datafile() (in module aietes.Tools)}

\begin{fulllineitems}
\phantomsection\label{index:aietes.Tools.get_latest_aietes_datafile}\pysiglinewithargsret{\code{aietes.Tools.}\bfcode{get\_latest\_aietes\_datafile}}{\emph{base\_dir=None}}{}~\begin{quote}\begin{description}
\item[{Parameters}] \leavevmode
\textbf{base\_dir} -- 

\item[{Returns}] \leavevmode
\begin{quote}\begin{description}
\item[{raise ValueError}] \leavevmode
\end{description}\end{quote}


\end{description}\end{quote}

\end{fulllineitems}

\index{grouper() (in module aietes.Tools)}

\begin{fulllineitems}
\phantomsection\label{index:aietes.Tools.grouper}\pysiglinewithargsret{\code{aietes.Tools.}\bfcode{grouper}}{\emph{data}}{}~\begin{quote}\begin{description}
\item[{Parameters}] \leavevmode
\textbf{data} -- 

\item[{Returns}] \leavevmode


\end{description}\end{quote}

\end{fulllineitems}

\index{is\_valid\_aietes\_datafile() (in module aietes.Tools)}

\begin{fulllineitems}
\phantomsection\label{index:aietes.Tools.is_valid_aietes_datafile}\pysiglinewithargsret{\code{aietes.Tools.}\bfcode{is\_valid\_aietes\_datafile}}{\emph{filename}}{}~\begin{quote}\begin{description}
\item[{Parameters}] \leavevmode
\textbf{filename} -- 

\item[{Returns}] \leavevmode


\end{description}\end{quote}

\end{fulllineitems}

\index{itersubclasses() (in module aietes.Tools)}

\begin{fulllineitems}
\phantomsection\label{index:aietes.Tools.itersubclasses}\pysiglinewithargsret{\code{aietes.Tools.}\bfcode{itersubclasses}}{\emph{cls}}{}
Generator over all subclasses of a given class, in depth first order.

\begin{Verbatim}[commandchars=\\\{\}]
\PYG{g+gp}{\PYGZgt{}\PYGZgt{}\PYGZgt{} }\PYG{n+nb}{list}\PYG{p}{(}\PYG{n}{itersubclasses}\PYG{p}{(}\PYG{n+nb}{int}\PYG{p}{)}\PYG{p}{)} \PYG{o}{==} \PYG{p}{[}\PYG{n+nb}{bool}\PYG{p}{]}
\PYG{g+go}{True}
\PYG{g+go}{:param cls:}
\PYG{g+go}{:param \PYGZus{}seen:}
\PYG{g+gp}{\PYGZgt{}\PYGZgt{}\PYGZgt{} }\PYG{k}{class} \PYG{n+nc}{A}\PYG{p}{(}\PYG{n+nb}{object}\PYG{p}{)}\PYG{p}{:} \PYG{k}{pass}
\PYG{g+gp}{\PYGZgt{}\PYGZgt{}\PYGZgt{} }\PYG{k}{class} \PYG{n+nc}{B}\PYG{p}{(}\PYG{n}{A}\PYG{p}{)}\PYG{p}{:} \PYG{k}{pass}
\PYG{g+gp}{\PYGZgt{}\PYGZgt{}\PYGZgt{} }\PYG{k}{class} \PYG{n+nc}{C}\PYG{p}{(}\PYG{n}{A}\PYG{p}{)}\PYG{p}{:} \PYG{k}{pass}
\PYG{g+gp}{\PYGZgt{}\PYGZgt{}\PYGZgt{} }\PYG{k}{class} \PYG{n+nc}{D}\PYG{p}{(}\PYG{n}{B}\PYG{p}{,}\PYG{n}{C}\PYG{p}{)}\PYG{p}{:} \PYG{k}{pass}
\PYG{g+gp}{\PYGZgt{}\PYGZgt{}\PYGZgt{} }\PYG{k}{class} \PYG{n+nc}{E}\PYG{p}{(}\PYG{n}{D}\PYG{p}{)}\PYG{p}{:} \PYG{k}{pass}
\PYG{g+go}{\PYGZgt{}\PYGZgt{}\PYGZgt{}}
\PYG{g+gp}{\PYGZgt{}\PYGZgt{}\PYGZgt{} }\PYG{k}{for} \PYG{n}{cls} \PYG{o+ow}{in} \PYG{n}{itersubclasses}\PYG{p}{(}\PYG{n}{A}\PYG{p}{)}\PYG{p}{:}
\PYG{g+gp}{... }    \PYG{k}{print}\PYG{p}{(}\PYG{n}{cls}\PYG{o}{.}\PYG{n}{\PYGZus{}\PYGZus{}name\PYGZus{}\PYGZus{}}\PYG{p}{)}
\PYG{g+go}{B}
\PYG{g+go}{D}
\PYG{g+go}{E}
\PYG{g+go}{C}
\PYG{g+go}{    \PYGZsh{} get ALL (new\PYGZhy{}style) classes currently defined}
\PYG{g+go}{    [cls.\PYGZus{}\PYGZus{}name\PYGZus{}\PYGZus{} for cls in itersubclasses(object)] }
\PYG{g+go}{[\PYGZsq{}type\PYGZsq{}, ...\PYGZsq{}tuple\PYGZsq{}, ...]}
\end{Verbatim}

\end{fulllineitems}

\index{kwarger() (in module aietes.Tools)}

\begin{fulllineitems}
\phantomsection\label{index:aietes.Tools.kwarger}\pysiglinewithargsret{\code{aietes.Tools.}\bfcode{kwarger}}{\emph{**kwargs}}{}~\begin{quote}\begin{description}
\item[{Parameters}] \leavevmode
\textbf{kwargs} -- 

\item[{Returns}] \leavevmode


\end{description}\end{quote}

\end{fulllineitems}

\index{linear2db() (in module aietes.Tools)}

\begin{fulllineitems}
\phantomsection\label{index:aietes.Tools.linear2db}\pysiglinewithargsret{\code{aietes.Tools.}\bfcode{linear2db}}{\emph{linear}}{}~\begin{quote}\begin{description}
\item[{Parameters}] \leavevmode
\textbf{linear} -- 

\item[{Returns}] \leavevmode


\end{description}\end{quote}

\end{fulllineitems}

\index{list\_functions() (in module aietes.Tools)}

\begin{fulllineitems}
\phantomsection\label{index:aietes.Tools.list_functions}\pysiglinewithargsret{\code{aietes.Tools.}\bfcode{list\_functions}}{\emph{module}}{}~\begin{quote}\begin{description}
\item[{Parameters}] \leavevmode
\textbf{module} -- 

\item[{Returns}] \leavevmode


\end{description}\end{quote}

\end{fulllineitems}

\index{listfix() (in module aietes.Tools)}

\begin{fulllineitems}
\phantomsection\label{index:aietes.Tools.listfix}\pysiglinewithargsret{\code{aietes.Tools.}\bfcode{listfix}}{\emph{list\_type}, \emph{value}}{}~\begin{quote}\begin{description}
\item[{Parameters}] \leavevmode\begin{itemize}
\item {} 
\textbf{list\_type} -- 

\item {} 
\textbf{value} -- 

\end{itemize}

\item[{Returns}] \leavevmode


\end{description}\end{quote}

\end{fulllineitems}

\index{literal\_eval\_walk() (in module aietes.Tools)}

\begin{fulllineitems}
\phantomsection\label{index:aietes.Tools.literal_eval_walk}\pysiglinewithargsret{\code{aietes.Tools.}\bfcode{literal\_eval\_walk}}{\emph{node}, \emph{tabs=0}}{}~\begin{quote}\begin{description}
\item[{Parameters}] \leavevmode\begin{itemize}
\item {} 
\textbf{node} -- 

\item {} 
\textbf{tabs} -- 

\end{itemize}

\end{description}\end{quote}

\end{fulllineitems}

\index{log\_level\_lookup() (in module aietes.Tools)}

\begin{fulllineitems}
\phantomsection\label{index:aietes.Tools.log_level_lookup}\pysiglinewithargsret{\code{aietes.Tools.}\bfcode{log\_level\_lookup}}{\emph{log\_level}}{}~\begin{quote}\begin{description}
\item[{Parameters}] \leavevmode
\textbf{log\_level} -- 

\item[{Returns}] \leavevmode


\end{description}\end{quote}

\end{fulllineitems}

\index{mag() (in module aietes.Tools)}

\begin{fulllineitems}
\phantomsection\label{index:aietes.Tools.mag}\pysiglinewithargsret{\code{aietes.Tools.}\bfcode{mag}}{\emph{vector}}{}
Return the magnitude of a given vector
\%timeit np.sqrt((uu1{[}0{]}-uu2{[}0{]})**2 +(uu1{[}1{]}-uu2{[}1{]})**2 +(uu1{[}2{]}-uu2{[}2{]})**2)-\textgreater{} 7.08us,
\%timeit np.linalg.norm(uu1-uu2) -\textgreater{} 11.7us.
:param vector:

\end{fulllineitems}

\index{memory() (in module aietes.Tools)}

\begin{fulllineitems}
\phantomsection\label{index:aietes.Tools.memory}\pysiglinewithargsret{\code{aietes.Tools.}\bfcode{memory}}{\emph{since=0.0}}{}
Return memory usage in Megabytes.
:param since:

\end{fulllineitems}

\index{mkcpickle() (in module aietes.Tools)}

\begin{fulllineitems}
\phantomsection\label{index:aietes.Tools.mkcpickle}\pysiglinewithargsret{\code{aietes.Tools.}\bfcode{mkcpickle}}{\emph{filename}, \emph{thing}}{}~\begin{quote}\begin{description}
\item[{Parameters}] \leavevmode\begin{itemize}
\item {} 
\textbf{filename} -- 

\item {} 
\textbf{thing} -- 

\end{itemize}

\item[{Returns}] \leavevmode


\end{description}\end{quote}

\end{fulllineitems}

\index{mkpickle() (in module aietes.Tools)}

\begin{fulllineitems}
\phantomsection\label{index:aietes.Tools.mkpickle}\pysiglinewithargsret{\code{aietes.Tools.}\bfcode{mkpickle}}{\emph{filename}, \emph{thing}}{}~\begin{quote}\begin{description}
\item[{Parameters}] \leavevmode\begin{itemize}
\item {} 
\textbf{filename} -- 

\item {} 
\textbf{thing} -- 

\end{itemize}

\item[{Returns}] \leavevmode


\end{description}\end{quote}

\end{fulllineitems}

\index{named\_log() (in module aietes.Tools)}

\begin{fulllineitems}
\phantomsection\label{index:aietes.Tools.named_log}\pysiglinewithargsret{\code{aietes.Tools.}\bfcode{named\_log}}{\emph{pos\_log}}{}~\begin{quote}\begin{description}
\item[{Parameters}] \leavevmode
\textbf{pos\_log} -- 

\item[{Returns}] \leavevmode


\end{description}\end{quote}

\end{fulllineitems}

\index{node\_ids() (in module aietes.Tools)}

\begin{fulllineitems}
\phantomsection\label{index:aietes.Tools.node_ids}\pysiglinewithargsret{\code{aietes.Tools.}\bfcode{node\_ids}}{\emph{pos\_log}}{}~\begin{quote}\begin{description}
\item[{Parameters}] \leavevmode
\textbf{pos\_log} -- 

\item[{Returns}] \leavevmode


\end{description}\end{quote}

\end{fulllineitems}

\index{notify\_desktop() (in module aietes.Tools)}

\begin{fulllineitems}
\phantomsection\label{index:aietes.Tools.notify_desktop}\pysiglinewithargsret{\code{aietes.Tools.}\bfcode{notify\_desktop}}{\emph{message}}{}
Thin Shim to notify when the simulations are complete, and not cry under no xsession
:param message:
:return:

\end{fulllineitems}

\index{object\_log() (in module aietes.Tools)}

\begin{fulllineitems}
\phantomsection\label{index:aietes.Tools.object_log}\pysiglinewithargsret{\code{aietes.Tools.}\bfcode{object\_log}}{\emph{pos\_log}, \emph{object\_id}}{}~\begin{quote}\begin{description}
\item[{Parameters}] \leavevmode\begin{itemize}
\item {} 
\textbf{pos\_log} -- 

\item {} 
\textbf{object\_id} -- 

\end{itemize}

\item[{Returns}] \leavevmode


\end{description}\end{quote}

\end{fulllineitems}

\index{random\_three\_vector() (in module aietes.Tools)}

\begin{fulllineitems}
\phantomsection\label{index:aietes.Tools.random_three_vector}\pysiglinewithargsret{\code{aietes.Tools.}\bfcode{random\_three\_vector}}{}{}
Generates a random 3D unit vector (direction) with a uniform spherical distribution
Algo from \href{http://stackoverflow.com/questions/5408276/python-uniform-spherical-distribution}{http://stackoverflow.com/questions/5408276/python-uniform-spherical-distribution}
:return:

\end{fulllineitems}

\index{random\_xy\_vector() (in module aietes.Tools)}

\begin{fulllineitems}
\phantomsection\label{index:aietes.Tools.random_xy_vector}\pysiglinewithargsret{\code{aietes.Tools.}\bfcode{random\_xy\_vector}}{}{}
Generates a random 2D vector in 3D space: (Planar random walk)
this is a horrible cheat but it works.
:return:

\end{fulllineitems}

\index{randomstr() (in module aietes.Tools)}

\begin{fulllineitems}
\phantomsection\label{index:aietes.Tools.randomstr}\pysiglinewithargsret{\code{aietes.Tools.}\bfcode{randomstr}}{\emph{length}}{}~\begin{quote}\begin{description}
\item[{Parameters}] \leavevmode
\textbf{length} -- 

\item[{Returns}] \leavevmode


\end{description}\end{quote}

\end{fulllineitems}

\index{range\_grouper() (in module aietes.Tools)}

\begin{fulllineitems}
\phantomsection\label{index:aietes.Tools.range_grouper}\pysiglinewithargsret{\code{aietes.Tools.}\bfcode{range\_grouper}}{\emph{data}}{}~\begin{quote}\begin{description}
\item[{Parameters}] \leavevmode
\textbf{data} -- 

\item[{Returns}] \leavevmode


\end{description}\end{quote}

\end{fulllineitems}

\index{records\_check() (in module aietes.Tools)}

\begin{fulllineitems}
\phantomsection\label{index:aietes.Tools.records_check}\pysiglinewithargsret{\code{aietes.Tools.}\bfcode{records\_check}}{\emph{pos\_log}}{}~\begin{quote}\begin{description}
\item[{Parameters}] \leavevmode
\textbf{pos\_log} -- 

\item[{Returns}] \leavevmode


\end{description}\end{quote}

\end{fulllineitems}

\index{resident() (in module aietes.Tools)}

\begin{fulllineitems}
\phantomsection\label{index:aietes.Tools.resident}\pysiglinewithargsret{\code{aietes.Tools.}\bfcode{resident}}{\emph{since=0.0}}{}
Return resident memory usage in Megabytes.
:param since:

\end{fulllineitems}

\index{results\_file() (in module aietes.Tools)}

\begin{fulllineitems}
\phantomsection\label{index:aietes.Tools.results_file}\pysiglinewithargsret{\code{aietes.Tools.}\bfcode{results\_file}}{\emph{proposed\_name}, \emph{results\_dir=None}}{}~\begin{quote}\begin{description}
\item[{Parameters}] \leavevmode\begin{itemize}
\item {} 
\textbf{proposed\_name} -- 

\item {} 
\textbf{results\_dir} -- 

\end{itemize}

\item[{Returns}] \leavevmode


\end{description}\end{quote}

\end{fulllineitems}

\index{sixvec() (in module aietes.Tools)}

\begin{fulllineitems}
\phantomsection\label{index:aietes.Tools.sixvec}\pysiglinewithargsret{\code{aietes.Tools.}\bfcode{sixvec}}{\emph{xyz}}{}~\begin{quote}\begin{description}
\item[{Parameters}] \leavevmode
\textbf{xyz} -- 

\item[{Returns}] \leavevmode


\end{description}\end{quote}

\end{fulllineitems}

\index{spherical\_distance() (in module aietes.Tools)}

\begin{fulllineitems}
\phantomsection\label{index:aietes.Tools.spherical_distance}\pysiglinewithargsret{\code{aietes.Tools.}\bfcode{spherical\_distance}}{\emph{sixvec\_a}, \emph{sixvec\_b}}{}~\begin{quote}\begin{description}
\item[{Parameters}] \leavevmode\begin{itemize}
\item {} 
\textbf{sixvec\_a} -- 

\item {} 
\textbf{sixvec\_b} -- 

\end{itemize}

\item[{Returns}] \leavevmode


\end{description}\end{quote}

\end{fulllineitems}

\index{stacksize() (in module aietes.Tools)}

\begin{fulllineitems}
\phantomsection\label{index:aietes.Tools.stacksize}\pysiglinewithargsret{\code{aietes.Tools.}\bfcode{stacksize}}{\emph{since=0.0}}{}
Return stack size in Megabytes.
:param since:

\end{fulllineitems}

\index{swapsize() (in module aietes.Tools)}

\begin{fulllineitems}
\phantomsection\label{index:aietes.Tools.swapsize}\pysiglinewithargsret{\code{aietes.Tools.}\bfcode{swapsize}}{\emph{since=0.0}}{}
Return memory usage in Megabytes.
:param since:

\end{fulllineitems}

\index{timeit() (in module aietes.Tools)}

\begin{fulllineitems}
\phantomsection\label{index:aietes.Tools.timeit}\pysiglinewithargsret{\code{aietes.Tools.}\bfcode{timeit}}{}{}~\begin{quote}\begin{description}
\item[{Returns}] \leavevmode


\end{description}\end{quote}

\end{fulllineitems}

\index{timestamp() (in module aietes.Tools)}

\begin{fulllineitems}
\phantomsection\label{index:aietes.Tools.timestamp}\pysiglinewithargsret{\code{aietes.Tools.}\bfcode{timestamp}}{}{}~\begin{quote}\begin{description}
\item[{Returns}] \leavevmode


\end{description}\end{quote}

\end{fulllineitems}

\index{try\_forever() (in module aietes.Tools)}

\begin{fulllineitems}
\phantomsection\label{index:aietes.Tools.try_forever}\pysiglinewithargsret{\code{aietes.Tools.}\bfcode{try\_forever}}{\emph{exceptions\_to\_catch}, \emph{fn}}{}~\begin{quote}\begin{description}
\item[{Parameters}] \leavevmode\begin{itemize}
\item {} 
\textbf{exceptions\_to\_catch} -- 

\item {} 
\textbf{fn} -- 

\end{itemize}

\item[{Returns}] \leavevmode


\end{description}\end{quote}

\end{fulllineitems}

\index{try\_x\_times() (in module aietes.Tools)}

\begin{fulllineitems}
\phantomsection\label{index:aietes.Tools.try_x_times}\pysiglinewithargsret{\code{aietes.Tools.}\bfcode{try\_x\_times}}{\emph{x}, \emph{exceptions\_to\_catch}, \emph{exception\_to\_raise}, \emph{fn}}{}~\begin{quote}\begin{description}
\item[{Parameters}] \leavevmode\begin{itemize}
\item {} 
\textbf{x} -- 

\item {} 
\textbf{exceptions\_to\_catch} -- 

\item {} 
\textbf{exception\_to\_raise} -- 

\item {} 
\textbf{fn} -- 

\end{itemize}

\item[{Returns}] \leavevmode
\begin{quote}\begin{description}
\item[{raise exception\_to\_raise}] \leavevmode
\end{description}\end{quote}


\end{description}\end{quote}

\end{fulllineitems}

\index{uncpickle() (in module aietes.Tools)}

\begin{fulllineitems}
\phantomsection\label{index:aietes.Tools.uncpickle}\pysiglinewithargsret{\code{aietes.Tools.}\bfcode{uncpickle}}{\emph{filename}}{}~\begin{quote}\begin{description}
\item[{Parameters}] \leavevmode
\textbf{filename} -- 

\item[{Returns}] \leavevmode


\end{description}\end{quote}

\end{fulllineitems}

\index{unext() (in module aietes.Tools)}

\begin{fulllineitems}
\phantomsection\label{index:aietes.Tools.unext}\pysiglinewithargsret{\code{aietes.Tools.}\bfcode{unext}}{\emph{filename}}{}~\begin{quote}\begin{description}
\item[{Parameters}] \leavevmode
\textbf{filename} -- 

\item[{Returns}] \leavevmode


\end{description}\end{quote}

\end{fulllineitems}

\index{unit() (in module aietes.Tools)}

\begin{fulllineitems}
\phantomsection\label{index:aietes.Tools.unit}\pysiglinewithargsret{\code{aietes.Tools.}\bfcode{unit}}{\emph{vector}}{}
Return the unit vector
:param vector:

\end{fulllineitems}

\index{unpickle() (in module aietes.Tools)}

\begin{fulllineitems}
\phantomsection\label{index:aietes.Tools.unpickle}\pysiglinewithargsret{\code{aietes.Tools.}\bfcode{unpickle}}{\emph{filename}}{}~\begin{quote}\begin{description}
\item[{Parameters}] \leavevmode
\textbf{filename} -- 

\item[{Returns}] \leavevmode


\end{description}\end{quote}

\end{fulllineitems}

\index{update\_dict() (in module aietes.Tools)}

\begin{fulllineitems}
\phantomsection\label{index:aietes.Tools.update_dict}\pysiglinewithargsret{\code{aietes.Tools.}\bfcode{update\_dict}}{\emph{d}, \emph{keys}, \emph{value}, \emph{safe=False}}{}~\begin{quote}\begin{description}
\item[{Parameters}] \leavevmode\begin{itemize}
\item {} 
\textbf{d} -- 

\item {} 
\textbf{keys} -- 

\item {} 
\textbf{value} -- 

\item {} 
\textbf{safe} -- 

\end{itemize}

\item[{Raises KeyError}] \leavevmode


\end{description}\end{quote}

\end{fulllineitems}

\index{validate\_config() (in module aietes.Tools)}

\begin{fulllineitems}
\phantomsection\label{index:aietes.Tools.validate_config}\pysiglinewithargsret{\code{aietes.Tools.}\bfcode{validate\_config}}{\emph{config=None}, \emph{final\_check=False}}{}
Generate valid confobj configuration information by interpolating a given config
file with the defaults
\begin{quote}\begin{description}
\item[{Parameters}] \leavevmode\begin{itemize}
\item {} 
\textbf{config} -- 

\item {} 
\textbf{final\_check} -- 

\end{itemize}

\end{description}\end{quote}

NOTE: This does not verify if any of the functionality requested in the config is THERE
Only that the config `makes sense' as requested.

I.e. does not check if particular modular behaviour exists or not.

\end{fulllineitems}

\phantomsection\label{index:module-bounos}\index{bounos (module)}\begin{itemize}
\item {} 
This file is part of the Aietes Framework (\href{https://github.com/andrewbolster/aietes}{https://github.com/andrewbolster/aietes})

\item {} 
\item {} \begin{enumerate}
\setcounter{enumi}{2}
\item {} 
Copyright 2013 Andrew Bolster (\href{http://andrewbolster.info/}{http://andrewbolster.info/}) and others.

\end{enumerate}

\item {} 
\item {} 
All rights reserved. This program and the accompanying materials

\item {} 
are made available under the terms of the Eclipse Public License v1.0

\item {} 
which accompanies this distribution, and is available at

\item {} 
\href{http://www.eclipse.org/legal/epl-v10.html}{http://www.eclipse.org/legal/epl-v10.html}

\item {} 
\item {} 
Contributors:

\item {} 
Andrew Bolster, Queen's University Belfast (-Aug 2013), University of Liverpool (Sept 2014-)

\end{itemize}
\index{BounosModel (class in bounos)}

\begin{fulllineitems}
\phantomsection\label{index:bounos.BounosModel}\pysiglinewithargsret{\strong{class }\code{bounos.}\bfcode{BounosModel}}{\emph{*args}, \emph{**kwargs}}{}~\begin{description}
\item[{BounosModel acts as an interactive superclass of DataPackage, designed for interactive}] \leavevmode
simulation/analysis

\item[{It is blankly initialised with no arguments and must be initialised by either interacting}] \leavevmode
with a simulation (update\_data\_from\_sim) or from an existing datafile (import\_datafile)

\end{description}
\index{update\_data\_from\_sim() (bounos.BounosModel method)}

\begin{fulllineitems}
\phantomsection\label{index:bounos.BounosModel.update_data_from_sim}\pysiglinewithargsret{\bfcode{update\_data\_from\_sim}}{\emph{p}, \emph{v}, \emph{names}, \emph{environment}, \emph{now}}{}
Call back function used by SimulationStep if doing real time simulation

Imports DataPackage data from the running simulation up to the requested time (self.t)

\end{fulllineitems}


\end{fulllineitems}

\index{add\_achievements() (in module bounos)}

\begin{fulllineitems}
\phantomsection\label{index:bounos.add_achievements}\pysiglinewithargsret{\code{bounos.}\bfcode{add\_achievements}}{\emph{ax}, \emph{d}, \emph{annotate\_achievements=False}}{}~\begin{quote}\begin{description}
\item[{Parameters}] \leavevmode\begin{itemize}
\item {} 
\textbf{ax} -- 

\item {} 
\textbf{d} -- 

\item {} 
\textbf{annotate\_achievements} -- 

\end{itemize}

\end{description}\end{quote}

\end{fulllineitems}

\index{custom\_fusion\_run() (in module bounos)}

\begin{fulllineitems}
\phantomsection\label{index:bounos.custom_fusion_run}\pysiglinewithargsret{\code{bounos.}\bfcode{custom\_fusion\_run}}{\emph{args}, \emph{data}, \emph{title}}{}~\begin{quote}\begin{description}
\item[{Parameters}] \leavevmode\begin{itemize}
\item {} 
\textbf{args} -- 

\item {} 
\textbf{data} -- 

\item {} 
\textbf{title} -- 

\end{itemize}

\end{description}\end{quote}

\end{fulllineitems}

\index{custom\_metric\_run() (in module bounos)}

\begin{fulllineitems}
\phantomsection\label{index:bounos.custom_metric_run}\pysiglinewithargsret{\code{bounos.}\bfcode{custom\_metric\_run}}{\emph{args}, \emph{data}, \emph{title}}{}~\begin{quote}\begin{description}
\item[{Parameters}] \leavevmode\begin{itemize}
\item {} 
\textbf{args} -- 

\item {} 
\textbf{data} -- 

\item {} 
\textbf{title} -- 

\end{itemize}

\end{description}\end{quote}

\end{fulllineitems}

\index{custom\_parser() (in module bounos)}

\begin{fulllineitems}
\phantomsection\label{index:bounos.custom_parser}\pysiglinewithargsret{\code{bounos.}\bfcode{custom\_parser}}{}{}~\begin{quote}\begin{description}
\item[{Returns}] \leavevmode


\end{description}\end{quote}

\end{fulllineitems}

\index{detect\_and\_identify() (in module bounos)}

\begin{fulllineitems}
\phantomsection\label{index:bounos.detect_and_identify}\pysiglinewithargsret{\code{bounos.}\bfcode{detect\_and\_identify}}{\emph{d}}{}~\begin{quote}\begin{description}
\item[{Parameters}] \leavevmode
\textbf{d} -- 

\item[{Returns}] \leavevmode


\end{description}\end{quote}

\end{fulllineitems}

\index{generate\_sources() (in module bounos)}

\begin{fulllineitems}
\phantomsection\label{index:bounos.generate_sources}\pysiglinewithargsret{\code{bounos.}\bfcode{generate\_sources}}{\emph{sources}, \emph{comms\_only=False}}{}
From a given list of DataPackage-able sources, yield a title/content tuple based on their d.title
\begin{quote}\begin{description}
\item[{Parameters}] \leavevmode\begin{itemize}
\item {} 
\textbf{sources} -- 

\item {} 
\textbf{comms\_only} -- Only return the comms substructure rather than the full datapackage

\end{itemize}

\item[{Return data}] \leavevmode
\end{description}\end{quote}

\end{fulllineitems}

\index{global\_adjust() (in module bounos)}

\begin{fulllineitems}
\phantomsection\label{index:bounos.global_adjust}\pysiglinewithargsret{\code{bounos.}\bfcode{global\_adjust}}{\emph{figure}, \emph{axes}, \emph{scale=2}}{}~\begin{quote}\begin{description}
\item[{Parameters}] \leavevmode\begin{itemize}
\item {} 
\textbf{figure} -- 

\item {} 
\textbf{axes} -- 

\item {} 
\textbf{scale} -- 

\end{itemize}

\end{description}\end{quote}
\begin{description}
\item[{General Figure adjustments:}] \leavevmode
Subplot-spacing adjustments
Figure sizing/scaling

\item[{Args:}] \leavevmode
figure(Figure): figure to be adjusted
scale(int/float): adjust figure to scale (optional:2)

\end{description}

\end{fulllineitems}

\index{load\_sources() (in module bounos)}

\begin{fulllineitems}
\phantomsection\label{index:bounos.load_sources}\pysiglinewithargsret{\code{bounos.}\bfcode{load\_sources}}{\emph{sources}, \emph{comms\_only=False}}{}
From a given list of DataPackage-able sources, parallelize their instantiation as a names dict based on their d.title
\begin{quote}\begin{description}
\item[{Parameters}] \leavevmode\begin{itemize}
\item {} 
\textbf{sources} -- 

\item {} 
\textbf{comms\_only} -- Only return the comms substructure rather than the full datapackage

\end{itemize}

\item[{Return data}] \leavevmode
\end{description}\end{quote}

\end{fulllineitems}

\index{main() (in module bounos)}

\begin{fulllineitems}
\phantomsection\label{index:bounos.main}\pysiglinewithargsret{\code{bounos.}\bfcode{main}}{}{}
Initial Entry Point; Does very little other that option parsing
Raises:
\begin{quote}

ValueError if graph selection doesn't make any sense
\end{quote}

\end{fulllineitems}

\index{multirun() (in module bounos)}

\begin{fulllineitems}
\phantomsection\label{index:bounos.multirun}\pysiglinewithargsret{\code{bounos.}\bfcode{multirun}}{\emph{args}, \emph{basedir='.'}}{}~\begin{quote}\begin{description}
\item[{Parameters}] \leavevmode\begin{itemize}
\item {} 
\textbf{basedir} -- 

\item {} 
\textbf{args} -- 

\end{itemize}

\item[{Returns}] \leavevmode


\end{description}\end{quote}

\end{fulllineitems}

\index{npz\_in\_dir() (in module bounos)}

\begin{fulllineitems}
\phantomsection\label{index:bounos.npz_in_dir}\pysiglinewithargsret{\code{bounos.}\bfcode{npz\_in\_dir}}{\emph{path}}{}
From a given dir, return all the NPZs
:param path:
:return sources:

\end{fulllineitems}

\index{plot\_detections() (in module bounos)}

\begin{fulllineitems}
\phantomsection\label{index:bounos.plot_detections}\pysiglinewithargsret{\code{bounos.}\bfcode{plot\_detections}}{\emph{ax}, \emph{metric}, \emph{orig\_data}, \emph{shade\_region=False}, \emph{real\_culprits=None}, \emph{good\_behaviour='Waypoint'}}{}
Plot Detection Overlay including False-positive analysis.

Will attempt heuristic analysis of `real' culprit from DataPackage behaviour records
\begin{quote}\begin{description}
\item[{Parameters}] \leavevmode\begin{itemize}
\item {} 
\textbf{ax} -- 

\item {} 
\textbf{metric} -- 

\item {} 
\textbf{orig\_data} -- 

\item {} 
\textbf{shade\_region} -- 

\item {} 
\textbf{real\_culprits} -- 

\item {} 
\textbf{good\_behaviour} -- 

\end{itemize}

\end{description}\end{quote}
\begin{description}
\item[{Args:}] \leavevmode
ax(axes): plot to operate on
metric(Metric): metric to use for detection
orig\_data(DataPackage): data used
shade\_region(bool): shade the detection region (optional:False)
real\_culprits(list): provide a list of culprits for false-positive testing (optional)
good\_behaviour(str): override the default good behaviour (optional: ``Waypoint'')

\end{description}

\end{fulllineitems}

\index{print\_analysis() (in module bounos)}

\begin{fulllineitems}
\phantomsection\label{index:bounos.print_analysis}\pysiglinewithargsret{\code{bounos.}\bfcode{print\_analysis}}{\emph{d}}{}~\begin{quote}\begin{description}
\item[{Parameters}] \leavevmode
\textbf{d} -- 

\item[{Returns}] \leavevmode


\end{description}\end{quote}

\end{fulllineitems}

\index{run\_detection\_fusion() (in module bounos)}

\begin{fulllineitems}
\phantomsection\label{index:bounos.run_detection_fusion}\pysiglinewithargsret{\code{bounos.}\bfcode{run\_detection\_fusion}}{\emph{data}, \emph{args=None}}{}~\begin{description}
\item[{Generate a trust fusion across available metrics, and plot both the metric deviations,}] \leavevmode
per-metric detections, and the trust fusion per node, per dataset

\end{description}
\begin{quote}\begin{description}
\item[{Parameters}] \leavevmode\begin{itemize}
\item {} 
\textbf{data} -- 

\item {} 
\textbf{args} -- 

\end{itemize}

\end{description}\end{quote}
\begin{description}
\item[{Args:}] \leavevmode
data(list of DataPackage): datasets to plot horizontally
args(argparse.NameSpace): formatting and option arguments (optional)

\end{description}

\end{fulllineitems}

\index{run\_metric\_comparison() (in module bounos)}

\begin{fulllineitems}
\phantomsection\label{index:bounos.run_metric_comparison}\pysiglinewithargsret{\code{bounos.}\bfcode{run\_metric\_comparison}}{\emph{data}, \emph{args=None}}{}~\begin{description}
\item[{Generate available metrics, and plot both the metric values,}] \leavevmode
per-metric detections, per dataset

\end{description}
\begin{quote}\begin{description}
\item[{Parameters}] \leavevmode\begin{itemize}
\item {} 
\textbf{data} -- 

\item {} 
\textbf{args} -- 

\end{itemize}

\end{description}\end{quote}
\begin{description}
\item[{Args:}] \leavevmode
data(list of DataPackage): datasets to plot horizontally
args(argparse.NameSpace): formatting and option arguments (optional)

\end{description}

\end{fulllineitems}

\index{run\_overlay() (in module bounos)}

\begin{fulllineitems}
\phantomsection\label{index:bounos.run_overlay}\pysiglinewithargsret{\code{bounos.}\bfcode{run\_overlay}}{\emph{data}, \emph{args=None}}{}~\begin{quote}\begin{description}
\item[{Parameters}] \leavevmode\begin{itemize}
\item {} 
\textbf{data} -- 

\item {} 
\textbf{args} -- 

\end{itemize}

\end{description}\end{quote}
\begin{description}
\item[{Args:}] \leavevmode
data(list of DataPackage): datasets to plot horizontally
args(argparse.NameSpace): formatting and option arguments (optional)

\end{description}

\end{fulllineitems}

\phantomsection\label{index:module-ephyra}\index{ephyra (module)}\begin{itemize}
\item {} 
This file is part of the Aietes Framework (\href{https://github.com/andrewbolster/aietes}{https://github.com/andrewbolster/aietes})

\item {} 
\item {} \begin{enumerate}
\setcounter{enumi}{2}
\item {} 
Copyright 2013 Andrew Bolster (\href{http://andrewbolster.info/}{http://andrewbolster.info/}) and others.

\end{enumerate}

\item {} 
\item {} 
All rights reserved. This program and the accompanying materials

\item {} 
are made available under the terms of the Eclipse Public License v1.0

\item {} 
which accompanies this distribution, and is available at

\item {} 
\href{http://www.eclipse.org/legal/epl-v10.html}{http://www.eclipse.org/legal/epl-v10.html}

\item {} 
\item {} 
Contributors:

\item {} 
Andrew Bolster, Queen's University Belfast (-Aug 2013), University of Liverpool (Sept 2014-)

\end{itemize}
\phantomsection\label{index:module-polybos}\index{polybos (module)}\begin{itemize}
\item {} 
This file is part of the Aietes Framework

\item {} 
(\href{https://github.com/andrewbolster/aietes}{https://github.com/andrewbolster/aietes})

\item {} 
\item {} \begin{enumerate}
\setcounter{enumi}{2}
\item {} 
Copyright 2013 Andrew Bolster (\href{http://andrewbolster.info/}{http://andrewbolster.info/}) and others.

\end{enumerate}

\item {} 
\item {} 
All rights reserved. This program and the accompanying materials

\item {} 
are made available under the terms of the Eclipse Public License v1.0

\item {} 
which accompanies this distribution, and is available at

\item {} 
\href{http://www.eclipse.org/legal/epl-v10.html}{http://www.eclipse.org/legal/epl-v10.html}

\item {} 
\item {} 
Contributors:

\item {} 
Andrew Bolster, Queen's University Belfast (-Aug 2013), University of Liverpool (Sept 2014-)

\end{itemize}
\index{ExperimentManager (class in polybos)}

\begin{fulllineitems}
\phantomsection\label{index:polybos.ExperimentManager}\pysiglinewithargsret{\strong{class }\code{polybos.}\bfcode{ExperimentManager}}{\emph{node\_count=None}, \emph{title=None}, \emph{parallel=False}, \emph{base\_config\_file=None}, \emph{base\_exp\_path=None}, \emph{*args}, \emph{**kwargs}}{}~\begin{quote}\begin{description}
\item[{Parameters}] \leavevmode\begin{itemize}
\item {} 
\textbf{node\_count} -- 

\item {} 
\textbf{title} -- 

\item {} 
\textbf{parallel} -- 

\item {} 
\textbf{base\_config\_file} -- 

\item {} 
\textbf{base\_exp\_path} -- 

\item {} 
\textbf{args} -- 

\item {} 
\textbf{kwargs} -- 

\end{itemize}

\end{description}\end{quote}
\index{add\_application\_variable\_scenario() (polybos.ExperimentManager method)}

\begin{fulllineitems}
\phantomsection\label{index:polybos.ExperimentManager.add_application_variable_scenario}\pysiglinewithargsret{\bfcode{add\_application\_variable\_scenario}}{\emph{variable}, \emph{value\_range}, \emph{title\_range=None}}{}
Add a scenario with a range of application/Node configuration values to the
experimental run

This \emph{UPDATES} the default nodes rather than adding custom ones
\begin{quote}\begin{description}
\item[{Parameters}] \leavevmode\begin{itemize}
\item {} 
\textbf{variable} -- 

\item {} 
\textbf{value\_range} -- 

\item {} 
\textbf{title\_range} -- 

\end{itemize}

\end{description}\end{quote}
\begin{description}
\item[{Args:}] \leavevmode
variable(str): mutable value description
value\_range(range or generator): values to be tested against.

\end{description}

\end{fulllineitems}

\index{add\_default\_scenario() (polybos.ExperimentManager method)}

\begin{fulllineitems}
\phantomsection\label{index:polybos.ExperimentManager.add_default_scenario}\pysiglinewithargsret{\bfcode{add\_default\_scenario}}{\emph{runcount=1}, \emph{title=None}}{}
Stick to the defaults
:param runcount:
:param title:

\end{fulllineitems}

\index{add\_minority\_n\_behaviour\_suite() (polybos.ExperimentManager method)}

\begin{fulllineitems}
\phantomsection\label{index:polybos.ExperimentManager.add_minority_n_behaviour_suite}\pysiglinewithargsret{\bfcode{add\_minority\_n\_behaviour\_suite}}{\emph{behaviour\_list}, \emph{n\_minority=1}, \emph{title='Behaviour'}}{}
Generate scenarios based on a list of `attacking' behaviours, i.e. minority behaviours
\begin{quote}\begin{description}
\item[{Parameters}] \leavevmode\begin{itemize}
\item {} 
\textbf{behaviour\_list} -- 

\item {} 
\textbf{n\_minority} -- 

\item {} 
\textbf{title} -- 

\end{itemize}

\end{description}\end{quote}
\begin{description}
\item[{Args:}] \leavevmode
behaviour\_list(list): minority behaviours
n\_minority(int): number of minority attackers in each scenario (optional)

\end{description}

\end{fulllineitems}

\index{add\_position\_scaling\_range() (polybos.ExperimentManager method)}

\begin{fulllineitems}
\phantomsection\label{index:polybos.ExperimentManager.add_position_scaling_range}\pysiglinewithargsret{\bfcode{add\_position\_scaling\_range}}{\emph{scale\_range}, \emph{title=None}, \emph{basis\_node\_name='n1'}, \emph{scale\_environment=True}, \emph{base\_scenario=None}}{}
Using the base\_config\_file, generate a range of scaled positions for nodes that are
manually set (i.e. operates only on the `initial\_position' value

ONLY DEALS IN 2D AND ASSUMES ALL Z-VALUES ARE THE SAME
:param scale\_range:
:param basis\_node\_name:
:param title:
:return:

\end{fulllineitems}

\index{add\_ratio\_scenarios() (polybos.ExperimentManager method)}

\begin{fulllineitems}
\phantomsection\label{index:polybos.ExperimentManager.add_ratio_scenarios}\pysiglinewithargsret{\bfcode{add\_ratio\_scenarios}}{\emph{badbehaviour}, \emph{goodbehaviour=None}}{}
Add scenarios based on a ratio of behaviours of identical nodes
\begin{description}
\item[{If goodbehaviour is not specified, then the default node configuration \emph{should} be used}] \leavevmode
for the remaining nodes

\end{description}
\begin{quote}\begin{description}
\item[{Parameters}] \leavevmode\begin{itemize}
\item {} 
\textbf{badbehaviour} -- 

\item {} 
\textbf{goodbehaviour} -- 

\end{itemize}

\end{description}\end{quote}
\begin{description}
\item[{Args:}] \leavevmode
badbehaviour(str):Aietes behaviour definition string (i.e. modulename)
goodbehaviour(str):Aietes behaviour definition string (i.e. modulename) (optional)

\end{description}

\end{fulllineitems}

\index{add\_variable\_2\_range\_scenarios() (polybos.ExperimentManager method)}

\begin{fulllineitems}
\phantomsection\label{index:polybos.ExperimentManager.add_variable_2_range_scenarios}\pysiglinewithargsret{\bfcode{add\_variable\_2\_range\_scenarios}}{\emph{v\_dict}}{}
Add a 2dim range of scenarios based on a dictionary of value ranges.
This generates a meshgrid and samples scenarios across the 2dim space
\begin{quote}\begin{description}
\item[{Parameters}] \leavevmode
\textbf{v\_dict} -- 

\end{description}\end{quote}
\begin{description}
\item[{Args:}] \leavevmode
v\_dict(dict):\{`variable':'value\_range', `variable':'value\_range'\}

\end{description}

\end{fulllineitems}

\index{add\_variable\_node\_scenario() (polybos.ExperimentManager method)}

\begin{fulllineitems}
\phantomsection\label{index:polybos.ExperimentManager.add_variable_node_scenario}\pysiglinewithargsret{\bfcode{add\_variable\_node\_scenario}}{\emph{node\_range}}{}
Add a scenario with a range of configuration values to the experimental run
\begin{quote}\begin{description}
\item[{Parameters}] \leavevmode
\textbf{node\_range} -- 

\end{description}\end{quote}
\begin{description}
\item[{Args:}] \leavevmode
variable(str): mutable value description
value\_range(range or generator): values to be tested against.

\end{description}

\end{fulllineitems}

\index{add\_variable\_range\_scenario() (polybos.ExperimentManager method)}

\begin{fulllineitems}
\phantomsection\label{index:polybos.ExperimentManager.add_variable_range_scenario}\pysiglinewithargsret{\bfcode{add\_variable\_range\_scenario}}{\emph{variable}, \emph{value\_range}, \emph{title\_range=None}}{}
Add a scenario with a range of configuration values to the experimental run
\begin{quote}\begin{description}
\item[{Parameters}] \leavevmode\begin{itemize}
\item {} 
\textbf{variable} -- 

\item {} 
\textbf{value\_range} -- 

\item {} 
\textbf{title\_range} -- 

\end{itemize}

\end{description}\end{quote}
\begin{description}
\item[{Args:}] \leavevmode
variable(str): mutable value description
value\_range(range or generator): values to be tested against.

\end{description}

\end{fulllineitems}

\index{dump\_analysis() (polybos.ExperimentManager method)}

\begin{fulllineitems}
\phantomsection\label{index:polybos.ExperimentManager.dump_analysis}\pysiglinewithargsret{\bfcode{dump\_analysis}}{}{}
Ignore actual simulation information, record trust analysis stats to a pickle

\end{fulllineitems}

\index{dump\_dataruns() (polybos.ExperimentManager method)}

\begin{fulllineitems}
\phantomsection\label{index:polybos.ExperimentManager.dump_dataruns}\pysiglinewithargsret{\bfcode{dump\_dataruns}}{}{}
Dump scenarios into the exp\_path directory

\end{fulllineitems}

\index{dump\_self() (polybos.ExperimentManager method)}

\begin{fulllineitems}
\phantomsection\label{index:polybos.ExperimentManager.dump_self}\pysiglinewithargsret{\bfcode{dump\_self}}{}{}
Attempt to dump the entire experiment state

\end{fulllineitems}

\index{generate\_simulation\_stats() (polybos.ExperimentManager method)}

\begin{fulllineitems}
\phantomsection\label{index:polybos.ExperimentManager.generate_simulation_stats}\pysiglinewithargsret{\bfcode{generate\_simulation\_stats}}{}{}~\begin{description}
\item[{Returns:}] \leavevmode
List of scenario stats (i.e. list of lists of run statistics dicts)

\end{description}

\end{fulllineitems}

\index{print\_stats() (polybos.ExperimentManager static method)}

\begin{fulllineitems}
\phantomsection\label{index:polybos.ExperimentManager.print_stats}\pysiglinewithargsret{\strong{static }\bfcode{print\_stats}}{\emph{experiment}, \emph{verbose=False}}{}~\begin{description}
\item[{Perform and print a range of summary experiment statistics including}] \leavevmode
Fleet Distance (sum of velocities),
Fleet Efficiency (Distance per time per node),
Stdev(INDA) (Proxy for fleet positional variability)
Stdev(INDD) (Proxy for fleet positional efficiency)
Max Achievement Count,
Percentage completion rate (how much of the fleet got the top count)

\end{description}
\begin{quote}\begin{description}
\item[{Parameters}] \leavevmode\begin{itemize}
\item {} 
\textbf{experiment} -- 

\item {} 
\textbf{verbose} -- 

\end{itemize}

\end{description}\end{quote}

\end{fulllineitems}

\index{run() (polybos.ExperimentManager method)}

\begin{fulllineitems}
\phantomsection\label{index:polybos.ExperimentManager.run}\pysiglinewithargsret{\bfcode{run}}{\emph{runtime=None}, \emph{runcount=None}, \emph{retain\_data=True}, \emph{queue=False}, \emph{**kwargs}}{}
Construct an execution environment and farm off simulation to scenarios
:param runtime:
:param runcount:
:param retain\_data:
:param kwargs:
Args:
\begin{quote}

runtime(int): Override simulation duration (normally inherited from config)
runcount(int): Number of repeated executions of this scenario; this value overrides the
\begin{quote}

value set on init
\end{quote}
\begin{description}
\item[{retain\_data(bool/str): Decides wether the scenario state data is maintained or allowed to be GC'd}] \leavevmode
can be one of {[}True,'file','additional\_only'{]}

\end{description}
\end{quote}

\end{fulllineitems}

\index{update\_all\_nodes() (polybos.ExperimentManager method)}

\begin{fulllineitems}
\phantomsection\label{index:polybos.ExperimentManager.update_all_nodes}\pysiglinewithargsret{\bfcode{update\_all\_nodes}}{\emph{config\_dict}}{}
Applys a behaviour (given as a string) to the experimental default for node generation
:param config\_dict:
Args:
\begin{quote}

behaviour(str): new default behaviour
\end{quote}

\end{fulllineitems}

\index{update\_default\_behaviour() (polybos.ExperimentManager method)}

\begin{fulllineitems}
\phantomsection\label{index:polybos.ExperimentManager.update_default_behaviour}\pysiglinewithargsret{\bfcode{update\_default\_behaviour}}{\emph{behaviour}}{}
Applys a behaviour (given as a string) to the experimental default for node generation
:param behaviour:
Args:
\begin{quote}

behaviour(str): new default behaviour
\end{quote}

\end{fulllineitems}

\index{update\_default\_node() (polybos.ExperimentManager method)}

\begin{fulllineitems}
\phantomsection\label{index:polybos.ExperimentManager.update_default_node}\pysiglinewithargsret{\bfcode{update\_default\_node}}{\emph{config\_dict}}{}
Applys a behaviour (given as a string) to the experimental default for node generation
:param config\_dict:
Args:
\begin{quote}

behaviour(str): new default behaviour
\end{quote}

\end{fulllineitems}

\index{update\_duration() (polybos.ExperimentManager method)}

\begin{fulllineitems}
\phantomsection\label{index:polybos.ExperimentManager.update_duration}\pysiglinewithargsret{\bfcode{update\_duration}}{\emph{tmax}}{}
Update the simulation time of currently configured scenarios
:param tmax:
Args:
\begin{quote}

tmax(int): update experiment simulation duration for all scenarios
\end{quote}

\end{fulllineitems}

\index{update\_environment() (polybos.ExperimentManager method)}

\begin{fulllineitems}
\phantomsection\label{index:polybos.ExperimentManager.update_environment}\pysiglinewithargsret{\bfcode{update\_environment}}{\emph{environment}}{}
Update the environment extent of currently configured scenarios
:param environment:
Args:
\begin{quote}

environment({[}int,int,int{]}): update experiment simulation environment for all scenarios
\end{quote}

\end{fulllineitems}

\index{update\_node\_counts() (polybos.ExperimentManager method)}

\begin{fulllineitems}
\phantomsection\label{index:polybos.ExperimentManager.update_node_counts}\pysiglinewithargsret{\bfcode{update\_node\_counts}}{\emph{new\_count}}{}
Updates the node-count across all scenarios
\begin{quote}\begin{description}
\item[{Parameters}] \leavevmode
\textbf{new\_count} -- 

\end{description}\end{quote}
\begin{description}
\item[{Args:}] \leavevmode
new\_count(int):new value to be used across scenarios

\end{description}

\end{fulllineitems}


\end{fulllineitems}

\index{PseudoScenario (class in polybos)}

\begin{fulllineitems}
\phantomsection\label{index:polybos.PseudoScenario}\pysigline{\strong{class }\code{polybos.}\bfcode{PseudoScenario}}
PseudoScenario(title, datarun)
\index{datarun (polybos.PseudoScenario attribute)}

\begin{fulllineitems}
\phantomsection\label{index:polybos.PseudoScenario.datarun}\pysigline{\bfcode{datarun}}
Alias for field number 1

\end{fulllineitems}

\index{title (polybos.PseudoScenario attribute)}

\begin{fulllineitems}
\phantomsection\label{index:polybos.PseudoScenario.title}\pysigline{\bfcode{title}}
Alias for field number 0

\end{fulllineitems}


\end{fulllineitems}

\index{RecoveredExperiment (class in polybos)}

\begin{fulllineitems}
\phantomsection\label{index:polybos.RecoveredExperiment}\pysiglinewithargsret{\strong{class }\code{polybos.}\bfcode{RecoveredExperiment}}{\emph{dirpath}}{}
SubClass to recover a partially executed experiment from an experiment directory.

Not guaranteed to work in any way what so ever.
:param dirpath:
:return:
\index{walk\_dir() (polybos.RecoveredExperiment class method)}

\begin{fulllineitems}
\phantomsection\label{index:polybos.RecoveredExperiment.walk_dir}\pysiglinewithargsret{\strong{classmethod }\bfcode{walk\_dir}}{\emph{path}}{}~\begin{quote}\begin{description}
\item[{Parameters}] \leavevmode
\textbf{path} -- 

\item[{Returns}] \leavevmode


\end{description}\end{quote}

\end{fulllineitems}


\end{fulllineitems}

\index{Scenario (class in polybos)}

\begin{fulllineitems}
\phantomsection\label{index:polybos.Scenario}\pysiglinewithargsret{\strong{class }\code{polybos.}\bfcode{Scenario}}{\emph{default\_config=None}, \emph{default\_config\_file=None}, \emph{runcount=1}, \emph{title=None}, \emph{*args}, \emph{**kwargs}}{}
Scenario Object

The Scenario Object deals with config management and passthrough, as well as some optional
execution characteristics. The purpose of this manager is to abstract as much as humanly
possible.
\index{add\_custom\_node() (polybos.Scenario method)}

\begin{fulllineitems}
\phantomsection\label{index:polybos.Scenario.add_custom_node}\pysiglinewithargsret{\bfcode{add\_custom\_node}}{\emph{variable\_map}, \emph{count=1}}{}
Adds a node to the scenario based on a dict of mutable key,values
:param variable\_map:
:param count:
Args:
\begin{quote}

variable\_map(dict): variables and values to be modified from the default
count(int): if set, creates count instances of the custom node
\end{quote}

\end{fulllineitems}

\index{add\_default\_node() (polybos.Scenario method)}

\begin{fulllineitems}
\phantomsection\label{index:polybos.Scenario.add_default_node}\pysiglinewithargsret{\bfcode{add\_default\_node}}{\emph{count=1}}{}
Adds a default node
:param count:
Args:
\begin{quote}

count(int): if set, creates count instances of the default node
\end{quote}

\end{fulllineitems}

\index{add\_node() (polybos.Scenario method)}

\begin{fulllineitems}
\phantomsection\label{index:polybos.Scenario.add_node}\pysiglinewithargsret{\bfcode{add\_node}}{\emph{node\_conf}, \emph{names=None}, \emph{count=1}}{}
Adds a node to the scenario based on a (hopefully valid) node configuration
:param node\_conf:
:param names:
:param count:
Args:
\begin{quote}

node\_conf(dict): Fully defined node config dict
names(list(str)): List of names for new nodes
count(int): if set, creates count instances of the node
\end{quote}
\begin{description}
\item[{Raises:}] \leavevmode
RuntimeError if name definition doesn't make sense

\end{description}

\end{fulllineitems}

\index{commit() (polybos.Scenario method)}

\begin{fulllineitems}
\phantomsection\label{index:polybos.Scenario.commit}\pysiglinewithargsret{\bfcode{commit}}{}{}
`Lock' the scenario, generating the final config, filling in any `empty' config sections
Raises:
\begin{quote}

RuntimeError: on attempting to commit and already committed scenario
\end{quote}

\end{fulllineitems}

\index{generate\_config() (polybos.Scenario method)}

\begin{fulllineitems}
\phantomsection\label{index:polybos.Scenario.generate_config}\pysiglinewithargsret{\bfcode{generate\_config}}{}{}
Generate a config dict from the current state of the planned scenario
Returns:
\begin{quote}

DataPackage compatible dict
\end{quote}

\end{fulllineitems}

\index{generate\_configobj() (polybos.Scenario method)}

\begin{fulllineitems}
\phantomsection\label{index:polybos.Scenario.generate_configobj}\pysiglinewithargsret{\bfcode{generate\_configobj}}{}{}
Generate a ConfigObj from the current state of the planned scenario
Returns:
\begin{quote}

DataPackage compatible ConfigObj
\end{quote}

\end{fulllineitems}

\index{generate\_run\_stats() (polybos.Scenario method)}

\begin{fulllineitems}
\phantomsection\label{index:polybos.Scenario.generate_run_stats}\pysiglinewithargsret{\bfcode{generate\_run\_stats}}{\emph{sim\_run\_dataset=None}}{}
Recieving a bounos.datapackage, generate relevant stats
This is nasty and I can't remember why I did it this way
:param sim\_run\_dataset:
Returns:
\begin{quote}

A list of dict's given from DataPackage.package\_statistics()
\end{quote}

\end{fulllineitems}

\index{get\_behaviour\_dict() (polybos.Scenario method)}

\begin{fulllineitems}
\phantomsection\label{index:polybos.Scenario.get_behaviour_dict}\pysiglinewithargsret{\bfcode{get\_behaviour\_dict}}{}{}
Generate and return a dict of currently configured behaviours wrt names of nodes
eg. \{`Waypoing':{[}'alpha','beta','gamma'{]},'Flock':{[}'omega'{]}\}
\begin{description}
\item[{Returns:}] \leavevmode
dict of behaviours associated with a list of node names

\end{description}

\end{fulllineitems}

\index{run() (polybos.Scenario method)}

\begin{fulllineitems}
\phantomsection\label{index:polybos.Scenario.run}\pysiglinewithargsret{\bfcode{run}}{\emph{runcount=None}, \emph{runtime=None}, \emph{*args}, \emph{**kwargs}}{}
Offload this to AIETES
:param runcount:
:param runtime:
:param args:
:param kwargs:
Args:

\end{fulllineitems}

\index{run\_parallel() (polybos.Scenario method)}

\begin{fulllineitems}
\phantomsection\label{index:polybos.Scenario.run_parallel}\pysiglinewithargsret{\bfcode{run\_parallel}}{\emph{runcount=None}, \emph{runtime=None}, \emph{queueing\_pool=False}, \emph{**kwargs}}{}
Offload this to AIETES multiprocessing queue
\begin{quote}\begin{description}
\item[{Parameters}] \leavevmode\begin{itemize}
\item {} 
\textbf{runtime} -- 

\item {} 
\textbf{kwargs} -- 

\item {} 
\textbf{runcount} -- 

\end{itemize}

\end{description}\end{quote}
\begin{description}
\item[{Args:}] \leavevmode\begin{description}
\item[{runcount(int): Number of repeated executions of this scenario; this value overrides the}] \leavevmode
value set on init

\end{description}

runtime(int): Override simulation duration (normally inherited from config)

\end{description}

\end{fulllineitems}

\index{set\_duration() (polybos.Scenario method)}

\begin{fulllineitems}
\phantomsection\label{index:polybos.Scenario.set_duration}\pysiglinewithargsret{\bfcode{set\_duration}}{\emph{tmax}}{}
Set the scenario simulation duration,
:param tmax:
Args:
\begin{quote}

tmax(int): New simulation time
\end{quote}

\end{fulllineitems}

\index{set\_environment() (polybos.Scenario method)}

\begin{fulllineitems}
\phantomsection\label{index:polybos.Scenario.set_environment}\pysiglinewithargsret{\bfcode{set\_environment}}{\emph{environment}}{}
Set the scenario simulation environment extent,
:param environment:
Args:
\begin{quote}

environment({[}int,int,int{]}): New simulation environment extent
\end{quote}

\end{fulllineitems}

\index{set\_node\_count() (polybos.Scenario method)}

\begin{fulllineitems}
\phantomsection\label{index:polybos.Scenario.set_node_count}\pysiglinewithargsret{\bfcode{set\_node\_count}}{\emph{count}}{}~\begin{description}
\item[{Set the scenario node count, but does not update the configuration (this is satisfied in}] \leavevmode
the commit method)

\end{description}
\begin{quote}\begin{description}
\item[{Parameters}] \leavevmode
\textbf{count} -- 

\end{description}\end{quote}
\begin{description}
\item[{Args:}] \leavevmode
count(int): New Node count

\end{description}

\end{fulllineitems}

\index{update\_default\_node() (polybos.Scenario method)}

\begin{fulllineitems}
\phantomsection\label{index:polybos.Scenario.update_default_node}\pysiglinewithargsret{\bfcode{update\_default\_node}}{\emph{variable}, \emph{value}}{}
Update the default node for the scenario.
:param variable:
:param value:
Args:
\begin{quote}

variable(str):The Variable to be modified (should me in the mutable map)
value: the value to set that variable to
\end{quote}
\begin{description}
\item[{Raises:}] \leavevmode
RuntimeError if attempting to modify after commit.

\end{description}

\end{fulllineitems}

\index{update\_node() (polybos.Scenario method)}

\begin{fulllineitems}
\phantomsection\label{index:polybos.Scenario.update_node}\pysiglinewithargsret{\bfcode{update\_node}}{\emph{node\_conf}, \emph{mutable}, \emph{value}}{}~\begin{description}
\item[{Used to update selected field mappings between scenario definition and}] \leavevmode
the scenario configspec, as defined in mutable\_node\_configs

\end{description}
\begin{quote}\begin{description}
\item[{Parameters}] \leavevmode\begin{itemize}
\item {} 
\textbf{node\_conf} -- 

\item {} 
\textbf{mutable} -- 

\item {} 
\textbf{value} -- 

\end{itemize}

\end{description}\end{quote}
\begin{description}
\item[{Args:}] \leavevmode
node\_conf(dict): current node configuration to be updated
mutable(str): a string describing the aspect to be changed, present in the mutable map
value(any): the mutable value to be set

\item[{Raises:}] \leavevmode
NotImplementedError: on invalid mutable key

\end{description}

\end{fulllineitems}

\index{write() (polybos.Scenario method)}

\begin{fulllineitems}
\phantomsection\label{index:polybos.Scenario.write}\pysiglinewithargsret{\bfcode{write}}{}{}
Dump the datafiles into a path but creating a folder with our name

\end{fulllineitems}


\end{fulllineitems}



\chapter{Indices and tables}
\label{index:indices-and-tables}\label{index:welcome-to-aietes-s-documentation}\begin{itemize}
\item {} 
\emph{genindex}

\item {} 
\emph{modindex}

\item {} 
\emph{search}

\end{itemize}


\renewcommand{\indexname}{Python Module Index}
\begin{theindex}
\def\bigletter#1{{\Large\sffamily#1}\nopagebreak\vspace{1mm}}
\bigletter{a}
\item {\texttt{aietes}}, \pageref{index:module-aietes}
\item {\texttt{aietes.Tools}}, \pageref{index:module-aietes.Tools}
\indexspace
\bigletter{b}
\item {\texttt{bounos}}, \pageref{index:module-bounos}
\indexspace
\bigletter{e}
\item {\texttt{ephyra}}, \pageref{index:module-ephyra}
\indexspace
\bigletter{p}
\item {\texttt{polybos}}, \pageref{index:module-polybos}
\end{theindex}

\renewcommand{\indexname}{Index}
\printindex
\end{document}
